Пусть задано вероятностное пространство $(\Omega, \mathfrak{F}, P)$.

\begin{definition}
Случайной величиной $\xi$ называется измеримая функция $\xi = \xi(\omega)$,
отображающая $\Omega$ в множество $\mathbb{R}$.
\end{definition}

\begin{definition}
Законом распределения СВ называется любое правило (таблица, функция), позволяющая находить вероятности
всех возможных событий, связаных с СВ.
\end{definition}

\begin{definition}
Функция $F_\xi(x) = P(\xi \leq x)$ называется функцией распределения СВ $\xi$.
\end{definition}

\subsection{Свойства функции распределения}
\begin{enumerate}
    \item dom$F(x) = \mathbb{R}$
    \item $0 \leq F(x) \leq 1$ $\forall x \in \mathbb{R}$
    \item $F(-\infty) = 0$, $F(+\infty) = 1$
    \item $P(a < \xi < b) = P(a \leq \xi < b) = P(a < \xi \leq b) =$ \\
          $P(a \leq \xi \leq b) = F(b) - F(a)$, $a \leq b$
    \item $F(x_1) \geq F(x_2)$, $x_1 > x_2$, т.е. $F(x)$ неубывает
    \item $$F(x) = \lim_{\epsilon \to 0}{F(x + \epsilon)}$$ при $\epsilon > 0$, т.е. $F(x)$ непрерывна справа
\end{enumerate}
