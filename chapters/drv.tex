\begin{definition}
Случайная величина называется дискретной, если множество ее возможных значений
конечно или счетно.
\end{definition}

\begin{remark}
Пусть $\xi \in X = {x_1, \ldots, x_n, \ldots}$. Тогда события
$$\{\xi = x_1\}, \ldots, \{\xi = x_n\}, \ldots$$
образуют Полную Группу Попарно Несовместных Событий(ПГПНС), и
    $$p_k = P(\xi = x_k)$$, а множество
    $$\{p_k\}_{k = 1}^{n (+\infty)}$$
полностью определяет закон распределения $\xi$.
    $$\sum_{k = 1}^{n (+\infty)} p_k = 1$$
\end{remark}

\begin{remark}
Очевидно, что $$F(x) = P(\xi \leq x) = \sum_{k : x_k \leq x}{} p_k$$
\end{remark}

\begin{definition}
Математическим ожиданием ДВС $\xi$ называется величина
    $$M[\xi] = \sum_{k}x_k p_k$$
\end{definition}

\begin{definition}
$n$-ым начальным моментом ДВС $\xi$ называется величина
    $$\nu_n[\xi] = M[\xi^n] = \sum_{k}{}x_k^np_k$$
\end{definition}

\begin{definition}
Дисперсией ДСВ $\xi$ называется величина
    $$D[\xi] = \nu_2[\xi] - (M[\xi])^2$$
\end{definition}

\subsection{Типовые дискретные случайные величины}
\subsubsection{Равномерное распределение на конечном множестве}
ДВС $\xi \sim R(X)$ ($\sim$ - распределена по закону),
$X = \{x_1, \ldots, x_n\}$, \\
$p_k = p = const$ $\forall k = 1, \ldots n.$

$$1 = \sum_{k = 1}^{n}p_k = np \Rightarrow p = \frac{1}{n}$$
$$M[\xi] = \sum_{k = 1}^{n}x_kp_k = \frac{x_1 + \ldots + x_n}{n}$$

\begin{example}
    Из полной колоды карт (52 листа) наугад достают
    по одной карте (без возвращения) до тех пор, пока
    не попадется дама пик. Сколько в среднем карт придется
    извлечь из колоды?

    \textbf{Решение:} Пусть $\xi$ - число извлеченных до успеха карт. Тогда
    $$\xi \sim R(X), X = {1, \ldots, 52}$$
    Вероятность извлечения дамы пик равна $\frac{1}{52}$.
    Среднее значение есть математическое ожидание этой случайной величины, поэтому
    $$M[\xi] = \frac{1 + 2 + \ldots + 52}{52} = \frac{(1 + 52) \cdot 52}{2 \cdot 52} = \frac{53}{2}$$
\end{example}

\subsubsection{Геометрическое распределение}
Будем проводить опыты по схеме Бернулли до первого успеха. $\xi$ - число
опытов до первого успеха. \\
Тогда $\xi \sim G(p)$, где $p$ - вероятность успеха в каждом опыте. Очевидно,
что $\xi \in X = \{1, 2, \ldots\}$.
$$p_k = P(\xi = k) = (1 - p)^{k - 1}p = q^{k - 1}p$$
$$M[\xi] = \sum_{k = 1}^{\infty}x_kp_k = \sum_{k = 1}^{\infty}kq^{k - 1}p = \frac{1}{p}$$
$$D[\xi] = \frac{q}{p^2}$$

\begin{example}
    Для поиска пропавшей экспедиции был выделен вертолет, который за один вылет
    обнаруживает экспедицию с  вероятностью $\frac{1}{3}$. Сколько в среднем вылетов
    нужно совершить для обнаружения экспедиции?

    \textbf{Решение:} В задаче $\xi$ - номер первого успеха (обнаружение экспедиции), т.е. $\xi \sim G(p)$
    . Среднее значение -
    математическое ожидание. Тогда среднее число вылетов есть
    $$M[\xi] = \frac{1}{p} = 3$$
\end{example}

\subsubsection{Распределение Бернулли}
$\xi \sim Bi(1, p)$, $\xi \in X = \{0, 1\}$, $p = P(\xi = 1)$
$$M[\xi] = p$$
$$D[\xi] = pq$$
Наиболее вероятное число успехов $k$: $p - q \leq k \leq 2p$

\begin{example}
    На испытательном стенде установлены четыре прибора.
    Каждый из них пройдет испытание с вероятностью
    0.9, 0.91, 0.92, 0.93 соответственно. Сколько в среднем приборов
    пройдет испытание?

    \textbf{Решение:} Введем индикаторную величину
    $$\xi_k = \begin{cases}
        1, & \textup{k-ый прибор прошел испытание} \\
        0, & \textup{иначе}
    \end{cases}
    \Rightarrow \xi_k \sim Bi(1, p_k)
    $$
    Тогда $\xi$, равная числу приборов, прошедших испытание равна
    $$\xi = \sum_{k = 1}^{4}\xi_k$$
    $$M[\xi] = \sum_{k = 1}^{4}M[\xi_k] = 0.9 + 0.91 + 0.92 + 0.93 = 3.66$$
\end{example}

\subsubsection{Биномиальное распределение}
Пусть ДСВ $\xi$ - число успехов в схеме Бернулли из $n$ опытов.
$$\xi \sim Bi(n, p)$$
$$\xi \in X \ \{0, 1, \ldots, n\}$$
$$p_k = P(\xi = k) = P_n(k) = C_n^kp^nq^{n - k}$$
$$M[\xi] = np$$
$$D[\xi] = npq$$
Наиболее вероятное число успехов $\overset{*}{\xi}$: $np - q \leq \overset{*}{\xi} \leq np + p$

\subsubsection{Распределение Пуассона}
$$\xi \in X = \{0, 1, \ldots, n\}$$
$$\xi \sim \Pi(a), a > 0$$
$$p_k = \pi_k(a) = \frac{a^k}{k!}e^{-a}$$
$$M[\xi] = a$$
$$\nu_2[\xi] = M[\xi^2] = a + a^2$$
$$D[\xi] = \nu_2[\xi] - (M[\xi])^2 = a^2 + a - a^2 = a$$
Наиболее вероятное значение $\overset{*}{\xi}$: $a - 1 \leq \overset{*}{\xi} \leq a$
\begin{remark}
    Распределение Пуассона является одной из основных математических моделей,
    которые рассматриваются в теории массового обслуживания.
\end{remark}

\begin{theorem}[Пуассона, усиленная]
    $$\forall \mathcal{M} \subset \{0, 1, \ldots, n\} \Rightarrow |\sum_{m \in \mathcal{M}}P_n(m)
        - \sum_{m \in \mathcal{M}}\Pi_m(np)| \leq np^2,
    $$
    где $P_n$ - вероятность по схеме Бернулли.
\end{theorem}

\begin{remark}
    С помощью этой теоремы при больших $n$ и малых $p$ распределение $Bi(n, p)$ можно
    приближенно заменить на $\Pi(np)$. Это важно для расчетов на ЭВМ.
\end{remark}

\begin{example}
    Число вызовов на телефонной станции за единицу времени можно рассматривать как случайную величину,
    распределенную по закону Пуассона с параметром 100. Каково наиболее вероятное значение этой величины.
    Чему равна вероятность этого значения?

    \textbf{Решение:} Наиболее вероятное значение:
    $$99 = 100 - 1 \leq \overset{*}{\xi} \leq 100,$$
    т.е. $\overset{*}{\xi} = 99, 100$. Вероятность этого значение равна
    $$
    P(\xi = 99) + P(\xi = 100) = \textup{вычислете сами!}
    $$
\end{example}
