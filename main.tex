\documentclass[12pt]{report}

\usepackage[T2A]{fontenc}
\usepackage{diagbox}
\usepackage{fullpage}
\usepackage{multicol,multirow}
\usepackage{tabularx}
\usepackage{ulem}
\usepackage{hyperref}
\usepackage[tocflat]{tocstyle}
\usetocstyle{standard}
\usepackage[utf8]{inputenc}
\usepackage[russian]{babel}
\usepackage{amsmath}
\usepackage{amssymb}
\usepackage{amsthm}

\theoremstyle{plain}
\newtheorem{theorem}{Th}
\newtheorem{lemma}{Лемма}
\newtheorem{corollary}{Следствие}

\theoremstyle{definition}
\newtheorem{definition}{Def}
\newtheorem{example}{Ex}
\newtheorem{remark}{Rem}

\hypersetup{
    colorlinks,
    citecolor=black,
    filecolor=black,
    linkcolor=black,
    urlcolor=black
    backref
}

\begin{document}

\begin{titlepage}
    \centering
    \vspace{5cm}
    \Large{\textbf{
        Пособие для подготовки\\
        к контрольной работе по ТВиМСу, \\
        составленное по конспектам лекций и семинаров, \\
        прочитанных Каном~Ю.~С. на потоке ПМИ \\
        8-го факультета МАИ в 2017 году}} \\
    \vspace{15cm}
    \textbf{Версия: 0.2}
\end{titlepage}

\thispagestyle{empty}
\begin{flushright}
    \Large{\textit{
        Посвящается \\ Неповторимой, Незабываемой, Прекраснейшей Математике, \\
        контрольной по ТВиМСу и всем тем, \\
        кто не был на лекциях и семинарах.
    }}
\end{flushright}
\newpage

\thispagestyle{empty}
\begin{center}
    \LARGE{\textbf{ВНИМАНИЕ!}}
\end{center}
\Large{
    Наборщик текста плохо разбирается в предмете, может опечататься, исказить смысл формулы и т.~д. и т.~п.
    Используйте на свой страх и риск. В том случае, если вы будете это использовать - следите за
    версией файла. Смена старшей цифры - исправлена фатальная теоретическая ошибка. НЕОБХОДИМО
    скачать последнюю версию файла и сверить изменения. Смена младшей цифры - исправление мелких опечаток или
    добавление новой главы.

    Ссылка на актуальную версию файла в README на \\ https://github.com/rkhomenko/TViMSTestMAI.git.

    Замечания, опечатки и т.д. - туда же.
}
\newpage

\tableofcontents

\chapter{Случайные величины}
\begin{flushright}
    \textit{
        В коридорах третьего корпуса стоят две студентки: \\
        одна машет перед носом у другой учебником по ТВиМСу: \\
        -- Ты читала эту жесть?! \\
        -- баян от Кана
    }
\end{flushright}
\section{Основные понятия}
Пусть задано вероятностное пространство $(\Omega, \mathfrak{F}, P)$.

\begin{definition}
Случайной величиной $\xi$ называется измеримая функция $\xi = \xi(\omega)$,
отображающая $\Omega$ в множество $\mathbb{R}$.
\end{definition}

\begin{definition}
Законом распределения СВ называется любое правило (таблица, функция), позволяющая находить вероятности
всех возможных событий, связаных с СВ.
\end{definition}

\begin{definition}
Функция $F_\xi(x) = P(\xi \leq x)$ называется функцией распределения СВ $\xi$.
\end{definition}

\subsection{Свойства функции распределения}
\begin{enumerate}
    \item dom$F(x) = \mathbb{R}$
    \item $0 \leq F(x) \leq 1$ $\forall x \in \mathbb{R}$
    \item $F(-\infty) = 0$, $F(+\infty) = 1$
    \item $P(a < \xi < b) = P(a \leq \xi < b) = P(a < \xi \leq b) =$ \\
          $P(a \leq \xi \leq b) = F(b) - F(a)$, $a \leq b$
    \item $F(x_1) \geq F(x_2)$, $x_1 > x_2$, т.е. $F(x)$ неубывает
    \item $$F(x) = \lim_{\epsilon \to 0}{F(x + \epsilon)}$$ при $\epsilon > 0$, т.е. $F(x)$ непрерывна справа
\end{enumerate}

\section{Свойства математического ожидания и дисперсии}
\subsection{Свойства математического ожидания}
\begin{enumerate}
    \item $M[a] = a$, где $a$ -  константа.
    \item $M[a\xi + b\eta] = aM[\xi] + bM[\eta]$
    \item $M[\xi\eta] = M[\xi]M[\eta]$, если $\xi,\eta$ - независимые СВ
\end{enumerate}

\subsection{Свойства дисперсии}
\begin{enumerate}
    \item $D[\xi] \geq 0$
    \item Если $a$ - константа, то $D[a] = 0$. Справедливо и обратное утверждение.
    \item $D[\xi + \eta] = D[\xi] + D[\eta] + 2cov(\xi, \eta)$ (см. случайные векторы)
    \item В частности, $D[\xi_1 + \ldots + \xi_n] = D[\xi_1] + \ldots + D[\xi_n]$ для любых
          независимых или некоррелированных СВ.
    \item $D[a\xi] = a^2D[\xi]$
    \item $D[-\xi] = D[\xi]$
    \item $D[\xi + a] = D[\xi]$
\end{enumerate}

\section{Дискретные случайные величины}
\begin{definition}
Случайная величина называется дискретной, если множество ее возможных значений
конечно или счетно.
\end{definition}

\begin{remark}
Пусть $\xi \in X = {x_1, \ldots, x_n, \ldots}$. Тогда события
$$\{\xi = x_1\}, \ldots, \{\xi = x_n\}, \ldots$$
образуют Полную Группу Попарно Несовместных Событий(ПГПНС), и
    $$p_k = P(\xi = x_k)$$, а множество
    $$\{p_k\}_{k = 1}^{n (+\infty)}$$
полностью определяет закон распределения $\xi$.
    $$\sum_{k = 1}^{n (+\infty)} p_k = 1$$
\end{remark}

\begin{remark}
Очевидно, что $$F(x) = P(\xi \leq x) = \sum_{k : x_k \leq x}{} p_k$$
\end{remark}

\begin{definition}
Математическим ожиданием ДВС $\xi$ называется величина
    $$M[\xi] = \sum_{k}x_k p_k$$
\end{definition}

\begin{definition}
$n$-ым начальным моментом ДВС $\xi$ называется величина
    $$\nu_n[\xi] = M[\xi^n] = \sum_{k}{}x_k^np_k$$
\end{definition}

\begin{definition}
Дисперсией ДСВ $\xi$ называется величина
    $$D[\xi] = \nu_2[\xi] - (M[\xi])^2$$
\end{definition}

\subsection{Типовые дискретные случайные величины}
\subsubsection{Равномерное распределение на конечном множестве}
ДВС $\xi \sim R(X)$ ($\sim$ - распределена по закону),
$X = \{x_1, \ldots, x_n\}$, \\
$p_k = p = const$ $\forall k = 1, \ldots n.$

$$1 = \sum_{k = 1}^{n}p_k = np \Rightarrow p = \frac{1}{n}$$
$$M[\xi] = \sum_{k = 1}^{n}x_kp_k = \frac{x_1 + \ldots + x_n}{n}$$

\begin{example}
    Из полной колоды карт (52 листа) наугад достают
    по одной карте (без возвращения) до тех пор, пока
    не попадется дама пик. Сколько в среднем карт придется
    извлечь из колоды?

    \textbf{Решение:} Пусть $\xi$ - число извлеченных до успеха карт. Тогда
    $$\xi \sim R(X), X = {1, \ldots, 52}$$
    Вероятность извлечения дамы пик равна $\frac{1}{52}$.
    Среднее значение есть математическое ожидание этой случайной величины, поэтому
    $$M[\xi] = \frac{1 + 2 + \ldots + 52}{52} = \frac{(1 + 52) \cdot 52}{2 \cdot 52} = \frac{53}{2}$$
\end{example}

\subsubsection{Геометрическое распределение}
Будем проводить опыты по схеме Бернулли до первого успеха. $\xi$ - число
опытов до первого успеха. \\
Тогда $\xi \sim G(p)$, где $p$ - вероятность успеха в каждом опыте. Очевидно,
что $\xi \in X = \{1, 2, \ldots\}$.
$$p_k = P(\xi = k) = (1 - p)^{k - 1}p = q^{k - 1}p$$
$$M[\xi] = \sum_{k = 1}^{\infty}x_kp_k = \sum_{k = 1}^{\infty}kq^{k - 1}p = \frac{1}{p}$$
$$D[\xi] = \frac{q}{p^2}$$

\begin{example}
    Для поиска пропавшей экспедиции был выделен вертолет, который за один вылет
    обнаруживает экспедицию с  вероятностью $\frac{1}{3}$. Сколько в среднем вылетов
    нужно совершить для обнаружения экспедиции?

    \textbf{Решение:} В задаче $\xi$ - номер первого успеха (обнаружение экспедиции), т.е. $\xi \sim G(p)$
    . Среднее значение -
    математическое ожидание. Тогда среднее число вылетов есть
    $$M[\xi] = \frac{1}{p} = 3$$
\end{example}

\subsubsection{Распределение Бернулли}
$\xi \sim Bi(1, p)$, $\xi \in X = \{0, 1\}$, $p = P(\xi = 1)$
$$M[\xi] = p$$
$$D[\xi] = pq$$
Наиболее вероятное число успехов $k$: $p - q \leq k \leq 2p$

\begin{example}
    На испытательном стенде установлены четыре прибора.
    Каждый из них пройдет испытание с вероятностью
    0.9, 0.91, 0.92, 0.93 соответственно. Сколько в среднем приборов
    пройдет испытание?

    \textbf{Решение:} Введем индикаторную величину
    $$\xi_k = \begin{cases}
        1, & \textup{k-ый прибор прошел испытание} \\
        0, & \textup{иначе}
    \end{cases}
    \Rightarrow \xi_k \sim Bi(1, p_k)
    $$
    Тогда $\xi$, равная числу приборов, прошедших испытание равна
    $$\xi = \sum_{k = 1}^{4}\xi_k$$
    $$M[\xi] = \sum_{k = 1}^{4}M[\xi_k] = 0.9 + 0.91 + 0.92 + 0.93 = 3.66$$
\end{example}

\subsubsection{Биномиальное распределение}
Пусть ДСВ $\xi$ - число успехов в схеме Бернулли из $n$ опытов.
$$\xi \sim Bi(n, p)$$
$$\xi \in X \ \{0, 1, \ldots, n\}$$
$$p_k = P(\xi = k) = P_n(k) = C_n^kp^nq^{n - k}$$
$$M[\xi] = np$$
$$D[\xi] = npq$$
Наиболее вероятное число успехов $\overset{*}{\xi}$: $np - q \leq \overset{*}{\xi} \leq np + p$

\subsubsection{Распределение Пуассона}
$$\xi \in X = \{0, 1, \ldots, n\}$$
$$\xi \sim \Pi(a), a > 0$$
$$p_k = \pi_k(a) = \frac{a^k}{k!}e^{-a}$$
$$M[\xi] = a$$
$$\nu_2[\xi] = M[\xi^2] = a + a^2$$
$$D[\xi] = \nu_2[\xi] - (M[\xi])^2 = a^2 + a - a^2 = a$$
Наиболее вероятное значение $\overset{*}{\xi}$: $a - 1 \leq \overset{*}{\xi} \leq a$
\begin{remark}
    Распределение Пуассона является одной из основных математических моделей,
    которые рассматриваются в теории массового обслуживания.
\end{remark}

\begin{theorem}[Пуассона, усиленная]
    $$\forall \mathcal{M} \subset \{0, 1, \ldots, n\} \Rightarrow |\sum_{m \in \mathcal{M}}P_n(m)
        - \sum_{m \in \mathcal{M}}\Pi_m(np)| \leq np^2,
    $$
    где $P_n$ - вероятность по схеме Бернулли.
\end{theorem}

\begin{remark}
    С помощью этой теоремы при больших $n$ и малых $p$ распределение $Bi(n, p)$ можно
    приближенно заменить на $\Pi(np)$. Это важно для расчетов на ЭВМ.
\end{remark}

\begin{example}
    Число вызовов на телефонной станции за единицу времени можно рассматривать как случайную величину,
    распределенную по закону Пуассона с параметром 100. Каково наиболее вероятное значение этой величины.
    Чему равна вероятность этого значения?

    \textbf{Решение:} Наиболее вероятное значение:
    $$99 = 100 - 1 \leq \overset{*}{\xi} \leq 100,$$
    т.е. $\overset{*}{\xi} = 99, 100$. Вероятность этого значение равна
    $$
    P(\xi = 99) + P(\xi = 100) = \textup{вычислете сами!}
    $$
\end{example}

\section{Непрерывные случайные величины}

\chapter{Случайные векторы}
\begin{flushright}
    \textit{
        Ноль семь, ноль восемь, меж берез и сосен? \\
        Пикник на обочине. \\
        -- баян от Кана
    }
\end{flushright}
\section{Двумерные случайные величины}
Пусть задано вероятностное пространство $(\Omega, \mathfrak{F}, P)$.

\begin{definition}
    Под случайным вектором размера $n$ будем понимать вектор
$$
\xi(\omega) = (\xi_1(\omega), \ldots, \xi_n(\omega))^T
$$
\end{definition}

\begin{remark}
    В дальнейшем будем рассматривать двумерный случай $(\xi,\eta)$
\end{remark}

\begin{definition}
    Функцией распределения вероятностей случайног вектора $(\xi, \eta)^T$ является
$$
F(x, y) = P(\xi \le x, \eta \le y)
$$
\end{definition}

\subsection{Свойства функции распределения}
\begin{enumerate}
    \item dom $F(x, y) = \mathbb{R}^2$
    \item $F(x, y) \in [0, 1]$
    \item $F(x, y)$ монотонно неубывает по каждому из аргументов
    \item $F(x, -\infty) = F(-\infty, y) = 0$ $\forall x,y \in \mathbb{R}$
    \item $F(x, +\infty) = F_\xi(x) = P(\xi \le x)$
    \item $F(+\infty, y) = F_\eta(y) = P(\eta \le y)$
    \item Если $F(x, y) = F_\xi(x)F_\eta(y)$, то $\xi,\eta$ - независимы
\end{enumerate}

\subsection{Числовые характеристики случайных векторов}
\begin{definition}
    Математическим ожиданием $\xi = (\xi_1, \ldots, \xi_n)^T$ называется вектор
$$
    M[\xi] = (M[\xi_1], \ldots, M[\xi_n])^T
$$
\end{definition}

\begin{remark}
У случайного вектора нет дисперсии, зато есть ковариационная матрица.
\end{remark}

\begin{definition}
Операция центрирования случайной величины $\circ$:
$$
\overset{\circ}{\xi} = \xi - M[\xi]
$$
\end{definition}

\begin{definition}
Операция нормирования случайной величины * определяется как
$$
\overset{*}{\xi} = \frac{\xi - M[\xi]}{D[\xi]},
$$
если у случайной величины существует дисперсия, и она отлична от нуля.
\end{definition}

\begin{definition}
Ковариацией случайных величин называется число
$$
cov(\xi, \eta) = M[\overset{\circ}{\xi}\overset{\circ}{\eta}],
$$
где $\xi,\eta$ - скаляры.
\end{definition}

\begin{definition}
Коэффициентом корреляции называется
$$
r_{\xi\eta} = cov(\overset{\circ}{\xi}, \overset{\circ}{\eta})
$$
\end{definition}

\begin{definition}
    Пусть $\xi = (\xi_1, \ldots, \xi_n)^T$, $k_{ij} = cov(\xi_i, \xi_j)$, тогда
    $K = (k_{ij}) = M[\overset{\circ}{\xi}\overset{\circ}{\xi}^T]$
    - ковариационная матрица,
    $R = (r_{\xi_i\xi_j})$ - корреляционная матрица.
\end{definition}

\subsection{Свойства числовых характеристик}
\begin{enumerate}
    \item $M[A\xi + B] = AM[\xi] + B$
    \item $cov(\xi, \xi) = D[\xi]$
    \item $cov(\xi, \eta) = cov(\eta, \xi)$
    \item $D[a^T\xi] = a^TKa$, где $a \in \mathbb{R}^n$
    \item Ковариационная матрица неотрицательно определена
    \item $cov(\xi, \eta) = M[\overset{\circ}{\xi}\eta] = M[\xi\overset{\circ}{\eta}]$
    \item $D[a\xi + b\eta] = a^2D[\xi] + b^2D[\eta] + 2abcov(\xi, \eta)$
    \item $D[\xi_1, \ldots, \xi_n] = \underset{i,j}{\Sigma}k_{ij}$
    \item $cov(\xi, \eta) = M[\xi\eta] - M[\xi]M[\eta]$
    \item $\xi, \eta$ - независимы $\Rightarrow$ $M[\xi\eta] = M[\xi]M[\eta]$
    \item $\xi, \eta$ - независимы, $\exists cov(\xi, \eta)$ $\Rightarrow cov(\xi, \eta) = 0$
    \item $r_{\xi\eta} = \frac{cov(\xi, \eta)}{\sigma_{\xi}\sigma_{\eta}}$, гдк $\sigma$ - СКО
    \item $\eta \ne a\xi + b, a \ne 0$ $\Rightarrow$ $r_{\xi\eta} = sign(a) =
        \begin{cases}
            1, & a > 0 \\
            -1, & a < 0
        \end{cases}
        $
    \item $|r_{\xi\eta}| = 1$
    \item $|cov(\xi, \eta)| \leq \sigma_\xi\sigma_\eta$
\end{enumerate}

\begin{definition}
    Случайный вектор называется дискретным, если он состоит из дискретных случайных величин.
\end{definition}
Пусть $X = {x_i}$ и $Y = {y_j}$ - множества значений ДСВ $\xi$ и $\eta$. Тогда $(x_i, y_j) \in X \times Y$,
$P(\xi = x_i, \eta = y_j) = p_{ij}$.

События ${\xi = x_i, \eta = y_j}$ образуют Полную Группу Попарно Несовместных Событий(ПГПНС).
Общая формула подсчета вероятности вида
$$
P(\xi,\eta \in T) = \underset{i,j~:~x_i, y_j \in T}{\Sigma}p_{ij}
$$

Зная $p_{ij}$ легко вычислить
$$
P(\xi = x_i) = p_{i\bullet} = \underset{j}{\Sigma}p_{ij}
$$
$$
P(\eta = y_j) = p_{\bullet j} = \underset{i}{\Sigma}p_{ij}
$$

\begin{theorem}
    ДВС $\xi,\eta$ - независимы $\Rightarrow$ $p_{ij} = p_{i\bullet}p_{\bullet j}$ $\forall i,j$
\end{theorem}

\begin{example}
    \begin{tabular}{|l|c|c|}
        \hline
        \diagbox{$\eta$}{$\xi$} & 0 & 1 \\ \hline
        $-1$ & $0.1$ & $0.2$ \\ \hline
        $0$ & $0.7$ & $0$ \\ \hline
    \end{tabular} \\
    Найти:
    \begin{enumerate}
        \item $P(|\xi| + |\eta| = 1)$
        \item Ковариационную матрицу
        \item Проверить независимость $\xi$  и $\eta$
    \end{enumerate}
    Решение:
    \begin{enumerate}
        \item $P(|\xi| + |\eta| = 1) = 0.1 + 0 = 0.1$
        \item
            $Law(\xi)$ \\
            \begin{tabular}{|l|c|c|} \hline
                $\xi$ & 0 & 1 \\ \hline
                $p = p_{i\bullet}$ & $0.1 + 0.7 = 0.8$ & $0.2 + 0 = 0.2$ \\ \hline
            \end{tabular}
            \\ $Law(\eta)$ \\
            \begin{tabular}{|l|c|c|} \hline
                $\eta$ & -1 & 0 \\ \hline
                $p = p_{\bullet j}$ & $0.1 + 0.2 = 0.3$ & $0.7 + 0 = 0.7$ \\ \hline
            \end{tabular}
            \\
            $$M[\xi] = 0.2$$
            $$D[\xi] = \nu_2[\xi] - (M[\xi])^2 = 0.2 - 0.2^2 = 0.16$$
            $$M[\eta] = -0.3$$
            $$D[\eta] = 0.21$$
            $$cov(\xi, \eta) = M[\xi\eta] - M[\xi]M[\eta] = -0.2 - 0.2 \cdot (-0.3) = -0.14$$
            $$M[\xi\eta] = \underset{i,j}{\Sigma}x_iy_jp_{ij} =
               0.1 \cdot 0 \cdot (-1) + 0.2 \cdot (-1) \cdot 1 + 0.7 \cdot 0 \cdot 0 + 0 \cdot 0 \cdot 1 = -0.2$$
             \[ \left( \begin{array}{cc}
                $$D[\xi]$$ & $$cov(\xi, \eta)$$ \\
                $$cov(\xi, \eta)$$ & $$D[\eta]$$ \end{array} \right) \] =
             \[ \left( \begin{array}{ccc}
                 $0.16$ & $-0.14$ \\
             $-0.14$ & $0.21$ \end{array} \right)\]
        \item Проверяем теорему о независимости ДСВ: $0.1 \ne 0.8 \cdot 0.3$ $\Rightarrow$ $\xi,\eta$ зависимы.
    \end{enumerate}
\end{example}


\section{Условные распределения}

\end{document}
